\documentclass[final]{beamer}
\usetheme{RJH}
\usepackage[orientation=landscape,size=a0,scale=1.4]{beamerposter}
\usepackage{tikz}
\usetikzlibrary{shapes,arrows,decorations.markings}
%\usepackage[orientation=landscape,height=1067mm,width=1524mm,scale=1.4,debug]{beamerposter}

\usepackage[absolute,overlay]{textpos}
\usepackage[numbers,compress]{natbib}
\usepackage{booktabs}
\usepackage{multirow}
\usepackage{array}
%\usepackage{wrapfig}
\usepackage{rotating}
\graphicspath{{Graphics/}}

% single line bib
% http://tex.stackexchange.com/a/5574
%\usepackage{paralist}
%\renewenvironment{thebibliography}[1]{%
%  \section*{\refname}%
%  \let\par\relax\let\newblock\relax%
%  \inparaenum[{[}1{]}]}{\endinparaenum}
%\renewcommand{\bibitem}[1]{\item}

\newcolumntype{C}[1]{>{\centering\arraybackslash} m{#1} }
\setlength{\TPHorizModule}{1cm}
\setlength{\TPVertModule}{1cm}
\newcommand{\subblock}[1]{\bigskip\textbf{#1}}

\title{Reliability of gamma activity during semantic integration}
\author{Jona Sassenhagen \and Phillip Alday}
\institute{University of Marburg}
\footer{}
%\footer{Live code at \url{https://github.com/jona-sassenhagen/phijona---gamma}}
\date{}

\begin{document}
\begin{frame}{} 	
\begin{textblock}{15}(1,0)
%\colorbox{white}{
\includegraphics[width=5in]{marburg-logo-blackwhite.eps}
%}
\end{textblock}


\begin{textblock}{45}(1,04.5)
\begin{block}{Motivation: Reproducibility of Results}
Recently, research on event-related spectral perturbations (ERSPs) has begun to focus on the gamma band of the human EEG. A number of recent studies have reported gamma effects for semantic integration (related to cloze probability in classical N400 paradigms) \cite{mellemfriedmanmedvedev2013a,wangzhubastiaansen2012a,penolazziangrillijob2009a,hagoort2008a,hagoorthaldbastiaansen2004a}, but the effects thus far have been far less consistent than previous findings in the lower frequency bands (cf.~\cite{bastiaansenhagoort2006a,heinetammhofmann2006a,rohmklimeschhaider2001a,davidsonindefrey2007a,roehmschlesewskybornkessel2004a}).
In line with recent concerns about the reliability of effects in the brain sciences \cite{vulharriswinkielman2009a,simmonsnelsonsimonsohn2011a,kilner2013a}, we performed a pseudo-jackknife replication meta-analysis of existing experimental data. 
% killner 2013
%\subblock{Title}
\end{block}

\begin{block}{Literature \hfill $p$-Values in Prior Research}
\begin{columns}
\column{0.05\textwidth}
\column{0.4\textwidth}
\tiny
\bibliographystyle{poster}
\bibliography{poster.bib}
\column{0.2\textwidth}
\column{0.7\textwidth}
\includegraphics[height=12.5cm]{kspcurve.png}
\end{columns}
\end{block}

\begin{block}{Method (Pseudo-Jackknife)}
\small
% Define block styles
% http://www.texample.net/tikz/examples/simple-flow-chart/
\tikzstyle{decision} = [diamond, draw, fill=blue!20, 
    text width=4.5em, text badly centered, inner sep=0pt,node distance=9.5cm]
\tikzstyle{block} = [rectangle, draw, fill=blue!20, 
    text width=5em, text centered, rounded corners, minimum height=4em, node distance = 7.5cm]
\tikzstyle{pole} = [rectangle, draw, fill=red!20, 
    text width=5em, text centered, rounded corners, minimum height=4em, node distance = 7.5cm]
\tikzstyle{line} = [draw, ultra thick, postaction={decorate}
									   ]
    
\begin{tikzpicture}[auto,decoration={markings,
								mark=at position 1 with {\arrow[scale=4,black]{latex'}};
								}]
    % Place nodes
    % example shapes: block, cloud, decision
    \node [pole] (init) {Start};
    \node [block,below of=init] (preproc) {Wavelet decomposition for all experiments};
    \node [block, below of=preproc] (foreach-outer) {For each experiment $i$ in set};
   	\node [block, below of=foreach-outer] (geteffects) {Find largest effect $e_i$ (time,space, frequency)};
	\node [block, right of=geteffects] (foreach-inner) {For each experiment $j$ in set};
	\node [block, right of=foreach-inner] (crosstest) {Test whether effect $e_i$ is significant in exp. $j$};
    \node [decision, right of=crosstest] (endfor-inner) {More exp. to crosstest?};
    \node [block, above of=crosstest] (forloop-inner) {Next exp. $j$};
    \node [decision, below of=geteffects] (endfor-outer) {More exp. to explore?};
    \node [block, left of=geteffects] (forloop-outer) {Next exp. $i$};
    \node [pole, below of=endfor-outer] (finish) {Stop}; 

    %\node [block, below of=] () {};
    % Draw edges
    \path [line] (init) -- (preproc);
    \path [line] (preproc) -- (foreach-outer);
    \path [line] (foreach-outer) -- (geteffects);
    \path [line] (geteffects) -> (foreach-inner);
    \path [line] (foreach-inner) -- (crosstest);
    \path [line] (crosstest) -- (endfor-inner);
    \path [line] (endfor-inner) |- node [near start] {yes} (forloop-inner);
    \path [line] (forloop-inner) -| (foreach-inner);
    \path [line] (endfor-inner) |- node {no} (endfor-outer);
    \path [line] (endfor-outer) -| node {yes} (forloop-outer);
    \path [line] (forloop-outer) |-  (foreach-outer);
	\path [line] (endfor-outer) -> node {no} (finish);
	
\end{tikzpicture}
\end{block}
\end{textblock}

\begin{textblock}{25}(20,38.5)
\begin{block}{Data}
\begin{itemize}
\item 11 Studies
\item 7 auditory, 4 visual presentation
\item 3 languages (German, Japanese, Welsh)
\item 300+ total subjects across studies
\end{itemize}
\end{block}
\end{textblock}

\begin{textblock}{69}(48.5,04.5)
\begin{block}{Replication}
\small \centering 
\begin{tabular}{c c c c}
\includegraphics{gamma01}      & \includegraphics{gamma02} & \includegraphics{gamma03}              & \multirow{4}{3cm}{\begin{sideways}\begin{minipage}{35cm}\small
HIGH, MED, LOW = high, medium, low cloze probability; \\ SEM = semantic judgment; COMP = comprehension task; PROBE = probe task; \\ ACC = acceptability judgment task; PASS = passive listening \\ DE = German; JA = Japanese; CY = Welsh 
\end{minipage}\end{sideways}} \\
DE / AUD / HIGH / SEM          & DE / AUD / MED / COMP     & DE / VIS / HIGH / PROBE \& ACC         & \\
\includegraphics{gamma04}      & \includegraphics{gamma05} & \includegraphics{gamma06}              & \\
DE / VIS / MED / PROBE \& ACC  & DE / VIS / MED / PASS     & DE / VIS / MED / PASS                  & \\
\includegraphics{gamma07}      & \includegraphics{gamma08} & \includegraphics{gamma09}              & \\
JA / VIS / MED / PROBE         & DE / AUD / MED / SEM      & DE / AUD / MED / SEM                   & \\
\includegraphics{gamma10}      & \includegraphics{gamma11} & \includegraphics[height=15cm]{reprate} & \\
DE / AUD /HIGH / SEM & CY / VIS / MED / PROBE \& ACC &  Summary -- All conditions \\
%German - visual & German auditory & English visual & English auditory \\ 
\end{tabular}

\tiny \raggedright Time-frequency plots (average of all channels) with the topography of the strongest effect in the alpha, theta and gamma bands for the \textsc{congruent} condition. Adjacent bar plots indicate statistical significance of the low-frequency (alpha and theta together) and high-frequency effect across all experiments. Final bar plot (lower right) for total statistical replication for all three conditions (\textsc{contrast}, \textsc{congruent}, \textsc{violation}).
\end{block}
%\end{textblock}

%\begin{textblock}{69}(48.5,76.5)
\begin{block}{Conclusion}
Unlike \textcolor{blue}{low-frequency effects}, \textcolor{red}{gamma effects} are not reliably elicited by the proposed experimental manipulation.
\end{block}
\end{textblock}

% \begin{textblock}{10}(113,69.5)
% \includegraphics[width=2in]{contactinfo-qrcode.png}
% \end{textblock}
% \begin{textblock}{10}(113,75.5)
%\includegraphics[width=2in]{code-qrcode.png}
% \end{textblock}

\end{frame}
\end{document}
