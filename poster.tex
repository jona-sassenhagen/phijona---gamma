\documentclass[final]{beamer}
\usetheme{RJH}
\usepackage[orientation=landscape,size=a0,scale=1.4]{beamerposter}
\usepackage{tikz}
\usetikzlibrary{shapes,arrows,decorations.markings}
%\usepackage[orientation=landscape,height=1067mm,width=1524mm,scale=1.4,debug]{beamerposter}

\usepackage[absolute,overlay]{textpos}
\usepackage[numbers]{natbib}
\usepackage{booktabs}
\usepackage{multirow}
\usepackage{array}
\usepackage{wrapfig}
\graphicspath{{Graphics/}}

\newcolumntype{C}[1]{>{\centering\arraybackslash} m{#1} }
\setlength{\TPHorizModule}{1cm}
\setlength{\TPVertModule}{1cm}
\newcommand{\subblock}[1]{\bigskip\textbf{#1}}

\title{Reliability of gamma activity during semantic integration}
\author{Jona Sassenhagen \and Phillip Alday}
\institute{University of Marburg}
\footer{Live code at \url{https://github.com/jona-sassenhagen/phijona---gamma}}
\date{}

\begin{document}
\begin{frame}{} 	
\begin{textblock}{15}(1,0)
%\colorbox{white}{
\includegraphics[width=5in]{marburg-logo-blackwhite.eps}
%}
\end{textblock}


\begin{textblock}{45}(1,04.5)
\begin{block}{Motivation: Reproducibility of Results}

%\subblock{Title}
\end{block}

\begin{block}{Prior Research}
\end{block}

\begin{block}{Method (Pseudo-Jackknife)}
% Define block styles
% http://www.texample.net/tikz/examples/simple-flow-chart/
\tikzstyle{decision} = [diamond, draw, fill=blue!20, 
    text width=4.5em, text badly centered, inner sep=0pt,node distance=10cm]
\tikzstyle{block} = [rectangle, draw, fill=blue!20, 
    text width=5em, text centered, rounded corners, minimum height=4em, node distance = 8cm]
\tikzstyle{pole} = [rectangle, draw, fill=red!20, 
    text width=5em, text centered, rounded corners, minimum height=4em, node distance = 8cm]
\tikzstyle{line} = [draw, ultra thick, postaction={decorate}
									   ]
%\tikzstyle{cloud} = [draw, ellipse,fill=red!20, node distance=3cm,minimum height=2em]
    
\begin{tikzpicture}[auto,decoration={markings,
								mark=at position 1 with {\arrow[scale=4,black]{latex'}};
								}]
    % Place nodes
    % example shapes: block, cloud, decision
    \node [pole] (init) {Start};
    \node [block,below of=init] (preproc) {Wavelet decomposition for all experiments};
    \node [block, below of=preproc] (foreach-outer) {For each experiment $i$ in set};
   	\node [block, below of=foreach-outer] (geteffects) {Find largest effect $e_i$ (time,space, frequency)};
	\node [block, right of=geteffects] (foreach-inner) {For each experiment $j$ in set};
	\node [block, right of=foreach-inner] (crosstest) {Test whether effect $e_i$ is significant in exp. $j$};
    \node [decision, right of=crosstest] (endfor-inner) {More exp. to crosstest?};
    \node [block, above of=crosstest] (forloop-inner) {Next exp. $j$};
    \node [decision, below of=geteffects] (endfor-outer) {More exp. to explore?};
    \node [block, left of=geteffects] (forloop-outer) {Next exp. $i$};
    \node [pole, below of=endfor-outer] (finish) {Stop}; 

    %\node [block, below of=] () {};
    % Draw edges
    \path [line] (init) -- (preproc);
    \path [line] (preproc) -- (foreach-outer);
    \path [line] (foreach-outer) -- (geteffects);
    \path [line] (geteffects) -> (foreach-inner);
    \path [line] (foreach-inner) -- (crosstest);
    \path [line] (crosstest) -- (endfor-inner);
    \path [line] (endfor-inner) |- node [near start] {yes} (forloop-inner);
    \path [line] (forloop-inner) -| (foreach-inner);
    \path [line] (endfor-inner) |- node {no} (endfor-outer);
    \path [line] (endfor-outer) -| node {yes} (forloop-outer);
    \path [line] (forloop-outer) |-  (foreach-outer);
	\path [line] (endfor-outer) -> node {no} (finish);
	
    %\path [line] (endfor-inner) |- (forloop-inner) -| (foreach-inner);
    %\path [line] (endfor-outer) -| (forloop-outer) |- (foreach-outer);

    %\path [line] (identify) -- (evaluate);
    %\path [line] (evaluate) -- (decide);
    %\path [line] (decide) -| node [near start] {yes} (update);
    %\path [line] (update) |- (identify);
    %\path [line] (decide) -- node {no}(stop);
    %\path [line,dashed] (expert) -- (init);
    %\path [line,dashed] (system) -- (init);
    %\path [line,dashed] (system) |- (evaluate);
\end{tikzpicture}
\end{block}

\end{textblock}

\begin{textblock}{57}(47.5,04.5)
\begin{block}{Replication}
\end{block}
\end{textblock}


\begin{textblock}{21.5}(90,68.7)
\begin{block}{Literature}
\tiny
%\bibliographystyle{pma-unsrt}
%\bibliography{poster.bib}
\end{block}
\end{textblock}

\begin{textblock}{10}(113,69.5)
%\includegraphics[width=2in]{contactinfo-qrcode.png}
\end{textblock}
\begin{textblock}{10}(113,75.5)
%\includegraphics[width=2in]{code-qrcode.png}
\end{textblock}

\end{frame}
\end{document}
